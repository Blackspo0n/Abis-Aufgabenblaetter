\section{Aufgabe 3 (Observer-Muster)}

\textbf{In einem Internet-Auktionshaus soll eine Funktion implementiert werden, die einem
Interessenten Bescheid gibt, wenn sich der Preis eines Artikels geändert hat. Implementieren
Sie diese Funktion (in Grundzügen, ähnlich wie in der Vorlesung) unter Verwendung des
Observer-Patterns.}

\textbf{Ausgabe der Konsole:}

\includegraphicsKeepAspectRatio{img2.png}{1}

\textbf{Folgende Klassen wurden definiert:}
\begin{lstlisting}[language=java, style=java, caption={Main.java},
label={lst:lst3}]
package observer;

import java.time.*;

public class Main {
	public static void main(String[] args) {
		Benutzer user_mar = new Benutzer(1, "Mario", "Kellner");
		Benutzer user_meh = new Benutzer(2, "Mehmet", "Tuefekci");
		Benutzer user_jul = new Benutzer(3, "Julian", "Kranen");
		
		Auktion laptop = new Auktion(1, "Ultra geiler Laptop", "Mega geiler Laptop mit Pentium 4 Prozessor", LocalDateTime.of(2017, Month.JULY, 05, 12, 30));
		
		user_meh.addTobeobachtungsListe(laptop);
		user_jul.addTobeobachtungsListe(laptop);

		laptop.setzeGebot(user_meh, 15.51f);
		
		laptop.setzeGebot(user_meh, 16f);
		
		laptop.setzeGebot(user_jul, 18.99f);
		
		laptop.setzeGebot(user_mar, 49.99f);
		
		laptop.setzeGebot(user_meh, 99.99f);	
	}
}
\end{lstlisting}

\begin{lstlisting}[language=java, style=java, caption={Gebot.java},
label={lst:lst4}]
package observer;

public class Gebot {
	public float preis;
	public Benutzer benutzer;
}
\end{lstlisting}

\begin{lstlisting}[language=java, style=java, caption={Benutzer.java},
label={lst:lst5}]
package observer;

import java.util.*;

public class Benutzer implements Observer {
	
	public Integer kundenNummer = 0;
	public String vorname = "";
	public String nachname = "";
	public List<Auktion> beobachtungsListe = new ArrayList<>();
	public List<Auktion> aktiveAuktionen = new ArrayList<>();
	
	public Benutzer(Integer kundenNummer, String vorname, String nachname) {
		kundenNummer = kundenNummer;
		vorname = vorname;
		nachname = nachname;
	}

	public void addToAktiveAuktionen(Auktion auk) {
		aktiveAuktionen.add(auk);
	}
	
	public void addTobeobachtungsListe(Auktion auk) {
		auk.addObserver(this);
		
		beobachtungsListe.add(auk);
	}
	
	@Override
	public void update(Observable arg0, Object arg1) {
		List<Gebot> gebote = (List<Gebot>)arg1;
		
		System.out.println("[" + Nachname + "] Gebot für " +((Auktion)arg0).ArikelName);
		System.out.println("[" + Nachname + "] Gebot von " + gebote.get(gebote.size()-1).benutzer.Nachname);
		System.out.println("[" + Nachname + "] Gebotener Preis " + gebote.get(gebote.size()-1).preis);
		System.out.println();
		
	}	
}
\end{lstlisting}

\begin{lstlisting}[language=java, style=java, caption={Auktion.java},
label={lst:lst6}]
package observer;

import java.time.LocalDateTime;
import java.util.*;

public class Auktion extends Observable {
	public Integer auktionsNr = 0;
	public String arikelName = "";
	public String beschreibung = "";
	
	public List<Gebot> gebote = new ArrayList<>();
	public LocalDateTime AuktionsEnde;
	
	public Auktion(Integer auktionsNr, String arikelName, String beschreibung, LocalDateTime auktionsEnde) {
		auktionsNr = auktionsNr;
		arikelName = arikelName;
		beschreibung = beschreibung;
		auktionsEnde = auktionsEnde;
		
		setChanged();
	}
	
	public void setzeGebot(Benutzer user, float preisInEuro) {
		if(AuktionsEnde.isAfter(LocalDateTime.now())) {
			if(gebote.isEmpty() || (preisInEuro > gebote.get(gebote.size()-1).preis)) {
				Gebot gebot = new Gebot();
				gebot.preis = preisInEuro;
				gebot.benutzer = user;
				
				gebote.add(gebot);
				user.addToAktiveAuktionen(this);
				
				setChanged();
				notifyObservers(gebote);
			}
		}
	}
}
\end{lstlisting}
\clearpage 