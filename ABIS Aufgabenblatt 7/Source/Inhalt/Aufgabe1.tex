\section{Aufgabe 1 (Strategiemuster)}
\textbf{Erweitern Sie das Beispiel zum Strategie-Muster aus der Vorlesung, indem Sie eine (Schrei-)
Stockente erzeugen, die ihr Quakverhalten von „Quaken“ auf „Schreien“ umschalten kann.
Welche neue(n) Klasse(n) müssen Sie implementieren?
Erstellen Sie den entsprechenden Quellcode.}

Da wir das Interface Quakverhalten haben und wir nur ein neues Quakverhalten
implementieren müssen, können wir das Interface \textbf{Quakverhalten} in eine
Klasse namens \gqq{Schreien} implementieren:

\begin{lstlisting}[language=java, style=java, caption={Schreien.java},
label={lst:lst1}]
public class Schreien implements QuakVerhalten {
    public void quaken() {
        System.out.println("Quarwwooooaaaaaarr!");
    }
}
\end{lstlisting}

Wir müssen anschließend nur die Funktion setQuakVerhalten mit unserem
Objekt aufrufen, damit wir das Quakverhalten zur Laufzeit ändern können:

\begin{lstlisting}[language=java, style=java, caption={EntenSimulator.java},
label={lst:lst2}]
	[...]
    public static void main(String[] args) {
        ArrayList<Ente> entenliste = new ArrayList<Ente>();
        entenliste.add(new Stockente());
        entenliste.add(new Moorente());
        entenliste.add(new Gummiente());
        entenliste.add(new Lockente());

        for (int x=0; x < entenliste.size(); x++)
            enteInAktion(entenliste.get(x));
        
        entenliste.get(0).setQuakVerhalten(new Schreien());
        
        enteInAktion(entenliste.get(0));
    }
	[...]
\end{lstlisting}
\clearpage

\textbf{Konsolenausgabe:}

\includegraphicsKeepAspectRatio{img3.png}{1}
\clearpage
