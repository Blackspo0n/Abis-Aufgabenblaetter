\section{Aufgabe 2 (Klassifikation)}

\subsection{Teiaufgabe A)}
\textbf{Suchen Sie in den drei in der Vorlesung vorgestellten Klassifikationssystemen nach einer
Batterie Monozelle 1,5V. Es handelt sich um eine Trockenbatterie („dry cell batterie“).
Geben Sie alle Stufen bis zum Ergebnis an.}

\subsubsection{UNSPSC}
Code: \textbf{26111705}

\begin{itemize}
  \item \textbf{26} Power Generation and Distribution Machinery and Accessories
  (Segment)
	\begin{itemize}
	  \item  \textbf{2611} Batteries and generators and kinetic power transmission
	  (Family)
	  \begin{itemize}
	    \item \textbf{261117} Batteries and cells and accessories (Class)
	    \begin{itemize}
	      \item  \textbf{26111705} Dry cell batteries (Commodity)
	      \end{itemize}
	   \end{itemize}
	\end{itemize}
\end{itemize}

\subsubsection{ETIM}

\begin{itemize} 
  \item  \textbf{EG000053} Batterie, Akku, Ladegerät, Netzkabel (Gruppe)
  \begin{itemize}
    \item \textbf{EC000356} Batterie - nicht wiederaufladbar (Klasse)
    \begin{itemize}
      \item  \textbf{EV000875} Mono (Merkmal)
      \end{itemize}
   \end{itemize}
\end{itemize}


\subsubsection{eCl@ss}
Code: \textbf{27-05-04-02}
\begin{itemize}
  \item  \textbf{27} Elektro-, Automatisierungs- und Prozessleittechnik
  \begin{itemize}
    \item \textbf{27-05} Akkumulator, Batterie
    \begin{itemize}
      \item  \textbf{27-05-04} Gerätebatterie
      	\begin{itemize}
	      \item  \textbf{27-05-04-02} Gerätebatterie Knopfzelle (nicht wieder aufladbar)
	    \end{itemize}
    \end{itemize}
   \end{itemize}
\end{itemize}
\clearpage

\subsection{Teiaufgabe B)}
\textbf{Vergleichen Sie Möglichkeiten und Grenzen der drei Klassifikationssysteme (kurz!)}

Mit den hier aufgezeigten Klassifizierungssysteme, hat ein Unternehmen die
Möglichkeit, ihr Produkt eindeutig klassifizierbar zu machen. Das hat den
Vorteil, dass keine Missverständnisse zwischen den IT-Systeme der Untnernehmen
auftreten können, sofern der jeweilige Standard beherrscht wird.

Es gibt trotz vieler Bemühungen zur Harmonisierung durchaus Unterschiede
zwischen den Klassifikationssystemen: so ist ETIM \zB nur in der Elektrobranche
vertreten. UNSPSC bietet offenbar keine Möglichkeit der Zertifizierung, deckt dafür aber
wirklich alle Bereiche ab. Teilweise gibt es Unterschiede in der Mitgliedschaft
bzw. in der Möglichkeit Änderungen einzureichen.

Das zeigt uns auf, dass ein Unternehmen, welches über die Klassifizierung ihrer
Produkte nachdenkt, erst einmal Rechercheaufwand betreiben müssen, um für sie
das optimalste System zu finden. Dabei sollte man sich jedoch stark nach der
Branche richten, in der man vertreten ist.

\clearpage 