\section{Aufgabe 1 (Schlüssel)}

\subsection{Teilaufgabe A)}
\textbf{Welchen Nutzen bietet die EAN/GTIN?}

Mit einer European Article Number(EAN) bzw. Global Trade Identifikation Number 
können Produkte global identifiziert werden. Die Nummer besteht aus 8 bzw. 13
Ziffern, von denen die ersten 2 oder 3 bzw. 7, 8 oder 9 Ziffern zentral durch die GS1-Gruppe verwaltet
und an Hersteller auf Antrag als Global Location Number vergeben werden.
\subsection{Teilaufgabe B)}
\textbf{Suchen Sie bei http://opengtindb.org und https://www.gepir.de Hersteller und ProduktInformationen
zu den Produkten mit den GTINs 4035800407007 und 4030300002851.}
\subsubsection{Super Dickmann's}

\begin{itemize}
  \item \textbf{Markenname} - {Dickmann}
  \item \textbf{Produkname} - {Super Dickmann's 9er, 6er Pack (6 x 250 g)}
  \item \textbf{Hersteller} - {August Storck KG}
  \item \textbf{Anschrift} - {Paulinenweg 12,
33790 Halle,
Deutschland}
\item \textbf{Basisnummer} - {4035800}
\end{itemize}

\subsubsection{Super Dickmann's}

\begin{itemize}
  \item \textbf{Markenname} - {Sallos}
  \item \textbf{Produkname} - {Sallos Bonbons, 5er Pack (5 x 150 g)}
  \item \textbf{Hersteller} - {Katjes Fassin GmbH + Co. KG}
  \item \textbf{Anschrift} - {Dechant-Sprünken-Str. 53-57,
46446 Emmerich,
Deutschland}
\item \textbf{Basisnummer} - {4030300}
\end{itemize}

\subsection{Teilaufgabe C)}
\textbf{Was ist die GLN und welches Ziel soll erreicht werden? Sie arbeiten bei der Edeka
Zentralverwaltung GmbH. Suchen Sie die zugehörige GLN. Wie viele Stellen hat die
Basisnummer für Ihr Unternehmen? Erzeugen Sie eine Fantasie-EAN/GTIN für Ihr
Unternehmen.}

\subsubsection{GLN}
Die Global Location Number bzw. Globale Lokationsnummer (GLN) identifiziert
global die volle Unternehmens- oder Betriebsbezeichnung sowie die Anschrift. 

Die GLN besteht aus 13 Ziffern. Die ersten drei Stellen enthalten das
Länderpräfix der GS1-Mitgliedsgesellschaft, zum Beispiel Deutschland 400–440.
Dann folgen 4 bis 6 Stellen, die zusammen mit der Ländernummer die 7- bis
9-stellige GS1-Basisnummer des Unternehmens bilden. In den folgenden 3 bis 5
Stellen bis zur Stelle 13 folgt ein individueller Nummerteil.

\subsubsection{Unser Unternehmen}
Unser Unternehmen, die \gqq{EDEKA Zentralverwaltung GmbH}, hat die GLN
\textbf{4312520000009}. Die GS1-Basisnummer, die wir für unsere GTIN benötigen,
ist 7 Zeichen lang und lautet \textbf{4312520}.

Fantasie-GLIN für neuen Blumenkohl: 4312520031239

\clearpage 