\section{Aufgabe 2 (Open Document und Office Open XML)}

\subsection{Teiaufgabe A)}
\textbf{Schauen Sie das Inhaltsverzeichnis zum OpenDocument-Standard unter
\url{http://docs.oasis-open.org/office/v1.2/OpenDocument-v1.2.html} an, um sich
einen Eindruck vom Aufbau und von der Komplexität des Standards zu verschaffen. Schauen
Sie sich ggf. einzelne Bestandteile genauer an. Open Document ist auch als
ISO-Standard aufgenommen. Wie lautet die entsprechende ISO Standard-Nummer?}

Die ISO-Nummer lautet:
\gqq{\textbf{ISO/IEC Standard 26300}}

\subsection{Teiaufgabe B)}
\textbf{Verschaffen Sie sich auch einen Überblick über Office Open XML unter
\href{http://www.ecma-international.org/publications/files/ECMA-ST/ECMA-376,
Fifth Edition, Part 1 - Fundamentals And Markup Language
Reference.zip}{http://www.ecma-international.org} Schauen Sie sich die
PDF-Datei an und vergleichen Sie die Komplexität mit Open Document und
beschreiben Sie das Ergebnis mit wenigen Sätzen.}

Das Open-Office-XML Format bietet eine wesentlich bessere nomierte und
strukturierte Normierung an. Der Office-Open-XML Standard bietet zudem
unteranderen Support für besondere Datentypen, wie \zB \gqq{.docm, .dot, .pptm},
an.

Im Vergleich zu Open Document ist die Implentieriung des Standards
vergleichweise komplex. Deshalb wird von den meisten Textverarbeitungsprogrammen
auch nur der OpenDocument Standard implementiert.

\subsection{Teiaufgabe C)}
\textbf{Unter welcher/welchen ISO-Nummer(n) ist Office Open XML normiert?}

Die ISO-Nummer lautet:
\gqq{\textbf{ISO/IEC 29500 Information technology – Office Open XML formats}}

\clearpage 

\subsection{Teiaufgabe D)}
\textbf{Erzeugen Sie mit Word und Excel aus Office 2010 oder höher ein Textdokument (docx)
bzw. eine Tabellenkalkulation (xlsx). Schauen Sie sich die erstellten Dateien mit einem
Dekomprimierer (pkzip, 7zip, o.ä.) an. Welche Dateien erhalten Sie (Screenshot eines
Verzeichnisses einfügen)?}

\includegraphicsKeepAspectRatio{docs.png}{1}


\subsection{Teiaufgabe E)}
\textbf{Warum gehen alle Hersteller zu XML-basierten Formaten über? Nennen Sie einige
Vorteile. Gibt es auch Nachteile?}

\tabelle{tab:tab1}{Vergleich}{Tabelle1}

\subsection{Teiaufgabe F)}
\textbf{Was spricht für den Einsatz von OpenDocument, was spricht für Office Open XML?}

\tabelle{tab:tab1}{Vergleich}{Tabelle2}

\subsection{Teiaufgabe G)}
\textbf{Als IT-Leiter eines Unternehmens mit 5000 MS-Office-Arbeitsplätzen auf veralteter SWBasis
stehen Sie vor der Frage, welche Strategie Sie in Zukunft fahren wollen. Bereiten
Sie Ihre Argumentation für die Geschäftsführung vor. Gehen Sie dabei vom aktuellen
Stand der Technik und Diskussion aus.}

Microsoft Office bietet eine benutzerfreundliche und standardisierte Oberfläche
an, die es dem Nutzer ermöglicht, dass jeweilige Programm innerhalb kürzester
Zeit optimal zu bedienen, siehe MS-Word und MS-Outlook.

Durch Open Office werden hohe Lizenskosten in 6 \bzw 7 stelliger Höhe, je nach
Umfang des Office-Pakets und Rabatte, eingespart. Zusätzlich kann Open Office
auch MS-Daten lesen und schreiben. Nicht nur MS-Daten, sondern auch
unterschiedliche bzw. besondere Datentypen können gelesen und geschrieben werden.

Als IT-Leiter unseres Unternehmens würde ich \gqq{Open Office} \gqq{Microsoft
Office} vorziehen, da die properitären Microsoft-Produkte durch frei verfügbare
opensource Produkte restlos ersetzt werden können.

\clearpage 