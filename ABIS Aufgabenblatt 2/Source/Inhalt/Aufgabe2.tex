\section{Aufgabe 2 (Open Document und Office Open XML)}

\subsection{Teiaufgabe A)}
\textbf{Schauen Sie das Inhaltsverzeichnis zum OpenDocument-Standard unter
\url{http://docs.oasis-open.org/office/v1.2/OpenDocument-v1.2.html} an, um sich
einen Eindruck vom Aufbau und von der Komplexität des Standards zu verschaffen. Schauen
Sie sich ggf. einzelne Bestandteile genauer an. Open Document ist auch als
ISO-Standard aufgenommen. Wie lautet die entsprechende ISO Standard-Nummer?}

Die ISO-Nummer lautet:
\gqq{\textbf{ISO/IEC Standard 26300}}

\subsection{Teiaufgabe B)}
\textbf{Verschaffen Sie sich auch einen Überblick über Office Open XML unter
\href{http://www.ecma-international.org/publications/files/ECMA-ST/ECMA-376,
Fifth Edition, Part 1 - Fundamentals And Markup Language
Reference.zip}{http://www.ecma-international.org} Schauen Sie sich die
PDF-Datei an und vergleichen Sie die Komplexität mit Open Document und
beschreiben Sie das Ergebnis mit wenigen Sätzen.}

\subsection{Teiaufgabe C)}
\textbf{Unter welcher/welchen ISO-Nummer(n) ist Office Open XML normiert?}

Die ISO-Nummer lautet:
\gqq{\textbf{ISO/IEC 29500 Information technology – Office Open XML formats}}

\clearpage 

\subsection{Teiaufgabe D)}
\textbf{Erzeugen Sie mit Word und Excel aus Office 2010 oder höher ein Textdokument (docx)
bzw. eine Tabellenkalkulation (xlsx). Schauen Sie sich die erstellten Dateien mit einem
Dekomprimierer (pkzip, 7zip, o.ä.) an. Welche Dateien erhalten Sie (Screenshot eines
Verzeichnisses einfügen)?}

\includegraphicsKeepAspectRatio{docs.png}{1}


\subsection{Teiaufgabe E)}
\textbf{Warum gehen alle Hersteller zu XML-basierten Formaten über? Nennen Sie einige
Vorteile. Gibt es auch Nachteile?}

Vorteile:
+ XML-basierte Formate können einfacher von anderen Anwendungen gelesen werden
+ Es stehen Standard-Mechanismen für das Lesen und Schreiben von XML zur Verfügung
+ Form und Inhalt werden getrennt

Nachteile:
- sehr „geschwätzig“

\subsection{Teiaufgabe F)}
\textbf{Was spricht für den Einsatz von OpenDocument, was spricht für Office Open XML?}

Open Document
+ Offener Standard
+ Mehrere, voneinander unabhängige Implementierungen
+ Erweiterbarkeit
Office Open XML
+ Marktmacht des Anbieters MS
+ Erfahrung des Anbieters
+ weit verbreitete Implementierung

\subsection{Teiaufgabe G)}
\textbf{Als IT-Leiter eines Unternehmens mit 5000 MS-Office-Arbeitsplätzen auf veralteter SWBasis
stehen Sie vor der Frage, welche Strategie Sie in Zukunft fahren wollen. Bereiten
Sie Ihre Argumentation für die Geschäftsführung vor. Gehen Sie dabei vom aktuellen
Stand der Technik und Diskussion aus.}


Beim Einsatz von Open Office sparen Sie Lizenzkosten in sechs- bis siebenstelliger Höhe
(je nach Umfang des Office-Pakets und eingeräumten Rabatten). Allerdings machen
Lizenzkosten erfahrungsgemäß nur einen relativ kleinen Anteil an den Kosten für einen PCArbeitsplatz
aus.

Open Office kann auch MS-Dateien lesen und schreiben. Wenn Sie häufig mit
unterschiedlichen Formaten konfrontiert sind, kann OO also eine Lösung sein.
Die Weiterentwicklung von OO ist allerdings momentan etwas unklar.

Als CIO gehen Sie persönlich mit dem Einsatz von MSO das geringere Risiko ein.


\clearpage 