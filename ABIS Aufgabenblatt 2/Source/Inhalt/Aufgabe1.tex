\section{Aufgabe 1 (Internet-Standards)}

\subsection{Teilaufgabe A)}
\textbf{Viele der für das Internet relevanten Standards sind in den so genannten RFCs (Request
for Comments) festgehalten (\url{http://www.rfc-editor.org/}). In RFC 2026 wird
der Standardisierungsprozess selbst beschrieben: Welche drei Ebenen (Reifegrade, Maturity
Levels) von Standards sind vorgesehen und wie sind diese charakterisiert?}

\subsubsection{Proposed Standard}

The entry-level maturity for the standards track is "Proposed Standard". A specific action by
the IESG is required to move a specification onto the standards track at the "Proposed
Standard" level. A Proposed Standard specification is generally stable, has resolved known
design choices, is believed to be well-understood, has received significant community
review, and appears to enjoy enough community interest to be considered valuable.
However, further experience might result in a change or even retraction of the specification
before it advances.

\subsubsection{Draft Standard}
A specification from which at least two independent and interoperable implementations
from different code bases have been developed, and for which sufficient successful
operational experience has been obtained, may be elevated to the "Draft Standard" level.
… Elevation to Draft Standard is a major advance in status, indicating a strong belief that
the specification is mature and will be useful.

\subsubsection{Internet Standard}
A specification for which significant implementation and successful operational
experience has been obtained may be elevated to the Internet Standard level. An Internet
Standard (which may simply be referred to as a Standard) is characterized by a high degree
of technical maturity and by a generally held belief that the specified protocol or service
provides significant benefit to the Internet community. A specification that reaches the
status of Standard is assigned a number in the STD series while retaining its RFC
number.

\clearpage 

\subsection{Teilaufgabe B)}
\textbf{Welche Standards werden durch die RFCs 791 und 793 beschrieben? Unter welcher
Standard-Nummer („STD“) sind diese Standards auch bekannt?}

\begin{itemize}
  \item RFC 791 = IP-Standard = STD 0005
  \item RFC 793 = TCP-Standard = STD 0007
\end{itemize}

\subsection{Teilaufgabe C)}
\textbf{Was wird durch RFC 1149 beschrieben?}

RFC 1149 = \gqq{Standard for the Transmission of IP Datagrams on Avian
Carriers}

Dieser Standard beschreibt den Transport von IP-Datenpaketen durch fliegende
Zubringer oder auch Brieftauben. Dies hat aber mit dem eigentlichen
IP-Protokoll nichts zutun.

Dieser Standard wurde am 01. April 1990 in den RFC Standard mit aufgenommen.
Dabei handelte es sich um einen Aprilscherz.
\clearpage 