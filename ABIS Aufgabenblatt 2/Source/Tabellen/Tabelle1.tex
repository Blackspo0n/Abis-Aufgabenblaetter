\newcolumntype{L}[1]{>{\raggedright\let\newline\\\arraybackslash\hspace{0pt}}m{#1}}
\newcolumntype{C}[1]{>{\centering\let\newline\\\arraybackslash\hspace{0pt}}m{#1}}
\newcolumntype{R}[1]{>{\raggedleft\let\newline\\\arraybackslash\hspace{0pt}}m{#1}}



\setlength{\extrarowheight}{2pt}
\setlength{\tabcolsep}{0.3em}
\begin{tabular}{|L{7cm}|L{7cm}|}
\rowcolor{heading}\color{white}\textbf{Vorteile} &
\color{white}\textbf{Nachteile}\\
\hline
XML standatisierte Formate können einfacher von anderen Anwendungen ausgelesen
werden & Sehr viel overhead \\ \hline 
XML Dokumente können mit Standardmittel gelesen und geschrieben werden &  \\
\hline

 Form und Inhalt werden getrennt &  \\ \hline

\end{tabular}
