\section{Aufgabe 3 (ebXML)}

\subsection{Teilaufgabe A)}
\textbf{Welche Rolle spielt die ebXML Registry?}

Die Registry hat 3 Aufgaben:

\begin{itemize}
  \item Die Vorlagen für verschiedene Geschäftsszenarien und Geschäftsprofile zu
speichern und diese im Bedarfsfall an Unternehmen, die ihre Systeme an den
ebXML-Standard anpassen wollen, auszuliefern.
  \item Details über die von einem Unternehmen unterstützen Geschäftsszenarien in
Form eines Geschäftsprofils zu speichern. 
  \item Anderen Unternehmen auf Anfrage mitteilen, welche Geschäftsszenarien von
einem Geschäftspartner unterstützt werden.
\end{itemize}

\subsection{Teilaufgabe B)}
\textbf{Wie muss ein Unternehmen verfahren, das Geschäftsprozesse elektronisch per
ebXML für andere Unternehmen verfügbar machen möchte? Beschreiben Sie die
erforderlichen Schritte kurz.}

Insgesamt sind mindestens drei schritte nötig, damit Unternehmen ihre
Geschäftsprozesse in ebXML abbilden können.

\subsubsection{Schritt 1: ebXML-Registry-Suche}
Firma A durchsucht das ebXML-Registry, um zu sehen, was bereits online
verfügbar ist. Im besten Fall kann Unternehmen A alle bestehenden
Geschäftsprozesse, Dokumente und Kernkomponenten wiederverwenden.

Ansonsten entwickelt Unternehmen A die fehlenden Dokumente eigentständig und
speichert diese in der ebXML Registry und macht sie für somit global zugänglich. 

\subsubsection{Schritt 2: ebXML-Business-Service-Interface}
Unternehmen A beschließt, ebXML für die elektronischen Prozesse zu verwenden. 

Eine ebXML-Business-Service-Interface (BSI) stellt die Verbindung zwischen dem
Unternehmen und ebXML her.

Das Unternehmen muss nun eine Collaboration Protocol Profile (CPP) erstellen,
die die unterstützten Geschäftsprozess-Funktionen, Einschränkungen, technische
Informationen und die Auswahl von Verschlüsselungsalgorithmen,
Verschlüsselungszertifikate und Transportprotokollen beschreibt. 

\subsubsection{Schritt 3: ebXML Registrierung}
Unternehmen A reicht seine CPP ebXML Registry ein. Von diesem Zeitpunkt an wird
das Unternehmen A öffentlich in der ebXML Registrierung aufgeführt und wird
kann von anderen Unternehmen gesucht werden.

\subsection{Teilaufgabe C)}
\textbf{Beschreiben Sie kurz die Rollen von Collaboration Protocol Profile (CPP) und
Collaboration Protocol Agreement (CPA) und beschreiben Sie kurz das
Zusammenspiel der beiden Protokolle.}

Ein Collaboration Protocol Profile (CPP) liefert alle notwendigen Informationen
darüber, wie ein bestimmter Handelspartner beabsichtigt elektronische Prozesse
ab zu wickeln.

Ein CPP definiert folgende Attribute eines Handelspartners:
\begin{itemize}
  \item Die Rolle diese innerhalb einer Collaboration
  \item Lieferkanäle und Transportprotokolle.
  \item Verpackung Art von Geschäftsdokumenten. 
  \item Sicherheitsbeschränkungen
  \item Per-Party-Konfiguration zu Geschäftsprozessspezifikationen. 
\end{itemize}

Die CPP ist in der ebXML Registry mit einem \textbf{Globally Unique Identifier
(GUID)} gespeichert. Geschäftspartner können andere CPP´s ebenfalls über den
Indetifier finden.

Da die Informationen innerhalb der CPP analysiert werden können, kann ein
potenzieller Kunde, auf Basis dieser Informationen, einschätzen, ob man sich
Geschäftsbeziehungen mit dem Unternehmen lohnen.
\clearpage 