\section{Aufgabe 3 (ebXML)}

\subsection{Teiaufgabe A)}
\textbf{Welche Rolle spielt die ebXML Registry?}
Die Registry hat 3 Aufgaben:

\begin{itemize}
  \item Die Vorlagen für verschiedene Geschäftsszenarien und Geschäftsprofile zu
speichern und diese im Bedarfsfall an Unternehmen, die ihre Systeme an den
ebXML-Standard anpassen wollen, auszuliefern.
  \item Details über die von einem Unternehmen unterstützen Geschäftsszenarien in
Form eines Geschäftsprofils zu speichern. 
  \item Anderen Unternehmen auf Anfrage mitteilen, welche Geschäftsszenarien von
einem Geschäftspartner unterstützt werden.
\end{itemize}

\subsection{Teiaufgabe B)}
\textbf{Wie muss ein Unternehmen verfahren, das Geschäftsprozesse elektronisch per
ebXML für andere Unternehmen verfügbar machen möchte? Beschreiben Sie die
erforderlichen Schritte kurz.}

\subsection{Teiaufgabe C)}
\textbf{Beschreiben Sie kurz die Rollen von Collaboration Protocol Profile (CPP) und
Collaboration Protocol Agreement (CPA) und beschreiben Sie kurz das
Zusammenspiel der beiden Protokolle.}

Das Collaboration Protocol Profile beschreibt die Merkmale und Fähigkeiten eines
Unternehmens bei der Ausführung von elektronischen Geschäftsprozessen. Dazu
gehören organisatorische Aspekte, wie der Name des Unternehmens und die
Nennung der Ansprechpartner und technische Aspekte, wie z.B. unterstützte
Transportprotokolle, Sicherheitsprotokolle, …

Durch die Abstimmung der CPPs zwischen zwei Unternehmen, die elektronische
Transaktionen abwickeln wollen und die Einigung auf die durchzuführenden
Transaktionen, entsteht ein Collaboration Protocol Agreement (CPA). Dieses stellt im
Prinzip die Schnittmenge der beiden CPPs bzw. eine Teilmenge davon dar.

\clearpage 