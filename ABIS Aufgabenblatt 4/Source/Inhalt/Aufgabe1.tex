\section{Aufgabe 1 (Transaktionsstandards)}
\textbf{Schauen Sie sich den Beispiel-Auftrag im OpenTrans-Format an. Sie finden die Datei auf
dem Moodle-Server (Datei \gqq{opentrans\_order\_1\_0\_example.xml}). Beantworten
Sie dazu die folgenden Fragen.}

\subsection{Teilaufgabe A)}
\textbf{Über welchen Mechanismus wird festgelegt, welche Tags in einem Auftrag erlaubt sind?
Wie heißen die zwei(!) Dateien, die den Aufbau festlegen?}

Der OpenTrans Standardd wird mittels DTD definiert. 
Die folgenden zwei Datein definieren diesen:
\begin{itemize}
  \item \textit{openbase\_1\_0.dtd}
  \item \textit{openTRANS\_ORDER\_1\_0.dtd}
\end{itemize}

\subsection{Teilaufgabe B)}
\textbf{Aus welchen Bestandteilen besteht ein Auftrag (nur Hauptkomponenten angeben)?}

Ein Auftrag besteht aus folgenden Bestandteilen:
\begin{itemize}
  \item ORDER\_HEADER
  \item ORDER\_ITEM\_LIST
  \item ORDER\_SUMMARY
\end{itemize}

Definiert sind diese Bestandteile in der Datei:
\textbf{openTRANS\_ORDER\_1\_0.dtd}
\clearpage 