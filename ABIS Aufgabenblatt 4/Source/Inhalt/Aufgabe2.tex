\section{Aufgabe 2 (Biztalk)}

\subsection{Teiaufgabe A)}
\textbf{Zu welchem Zweck kann ein Unternehmen den Biztalk-Server einsetzen (kurze
Beschreibung)?}

\subsection{Teiaufgabe B)}
\textbf{Welche Rolle spielt die \gqq{Messaging}-Komponente beim Biztalk-Server?
Machen Sie die Rolle am Beispiel eines eingehenden Auftrags mit Bonitätsprüfung des Kunden
deutlich.}

Die Messaging-Komponente nimmt Nachrichten von der Umwelt des Biztalk-Servers
entgegen bzw. gibt Nachrichten an diese weiter. Bei einem eingehenden Auftrag wird
von der Messaging-Komponente die wesentlichen Bestandteile aus dem Auftrag
geholt. Über einen Workflow wird die Erstellung von zwei Ausgangsnachrichten
angestoßen: Eine Nachricht zur Bonitätsprüfung des Kunden und eine Nachricht zur
Verfügbarkeitsprüfung der gewünschten Artikel. Beide Nachrichten werden wieder
über die Messaging-Komponente an die entsprechenden Anwendungen geschickt.

\subsection{Teiaufgabe C)}
\textbf{Welche Aufgabe hat die \gqq{Orchestration}-Komponente beim
Biztalk-Server?}

Die Orchestration-Komponente regelt das Zusammenspiel der Biztalk-Komponenten.
Zur Unterstützung eines Geschäftsprozesses werden hier die entsprechenden
Dienstleistungen aus den anderen Komponenten in einem Workflow angeordnet.

\clearpage 