\section{Aufgabe 2 (Biztalk)}

\subsection{Teilaufgabe A)}
\textbf{Zu welchem Zweck kann ein Unternehmen den Biztalk-Server einsetzen (kurze
Beschreibung)?}

Der Zweck eines Microsoft BizTalk Server ist es Daten zwischen zwei oder
mehreren Partnern auszutauschen. Dabei ist es in erster Linie unerheblich, um
welchen Typ es sich dabei handelt. BizTalk kann insich verschiedene Systeme
verbinden, Daten und Nachrichten mit Partnern auszutauschen und interne und
externe Prozesse steuern.

Einige Anwendungsfelder sind dabei folgende:
\begin{itemize}
  \item Abwicklung von Bestellprozessen mit externen Partnern über Edifact, XML,
  CSV
  \item Steuerung und Überwachung von Geschäftsprozessen mittels der BizTalk
  Workflow Engine
  \item Integration von Web-Portalen und Online-Shops
\end{itemize}

\subsection{Teilaufgabe B)}
\textbf{Welche Rolle spielt die \gqq{Messaging}-Komponente beim Biztalk-Server?
Machen Sie die Rolle am Beispiel eines eingehenden Auftrags mit Bonitätsprüfung des Kunden
deutlich.}

Die Kernkomponente des BizTalk Server ist die Messaging Engine zur
Nachrichtenverarbeitung. 

Diese bildet den gesamten Prozess der Nachrichtenverarbeitung ab: Vom Empfang
eingehender Nachrichten und der Ermittlung des Nachrichtenformats über die
Nachrichtenauswertung bis hin zur Auslieferung an den Nachrichtenempfänger und
der abschließlichden Nachrichtenverfolgung.

Eine Nachricht zur Bonitätsprüfung des Kunden und eine Nachricht zur
Verfügbarkeitsprüfung der gewünschten Artikel. Beide Nachrichten werden wieder
über die Messaging-Komponente an die entsprechenden Anwendungen geschickt.

\clearpage
\subsection{Teilaufgabe C)}
\textbf{Welche Aufgabe hat die \gqq{Orchestration}-Komponente beim
Biztalk-Server?}

Die Orchestration-Komponente stellt sog. \textbf{Orchestrations} zur Verfügung.
Die Orchestrations stellen die Logik eines Programms dar und werden über ein
grafisches Flussdiagramm entwickelt.

\includegraphicsKeepAspectRatio{BizTalkServer_Overview.png}{1.0}
\clearpage