\section{Aufgabe 1}

\subsection{Teilaufgabe A)}
\textbf{Nennen und erläutern Sie kurz die 10 Charakteristika betrieblicher
Anwendungen.}

\subsubsection{Strukturierte Daten}
Bei diesen Daten handelt es sich um Daten die einer festen Struktur folgen.
Ein bestimmtes Element in einer Zusammenstellung von Daten ist mit einer
festgelegten Bedeutung versehen, die sich in der Regel auch darin äußert, dass
dass Element einen Namen erhält.

Ein Beispiel dafür wäre die Lieferanten Nummer in der Lieferanten Datenbank.

\subsubsection{Stamm- und Bewegungsdaten}
Die typischen Stammdaten sind Artikelstammdaten, aber auch Kundendaten und
Lieferantendaten fallen in dieses Schema. Diese Daten werden einmalig
hinterlegt und sind meistens von einem Geschäftsprozess losgelöst zu sehen, da
in diesen keine zeitliche Betrachtung stattfindet.

Bewegungsdaten erfassen, wie der Name schon sagt, bewegungen meist über einen
bestimmten Zeitraum. Meistens kann man konkreten Bewegungsdaten konkrete
geschäftsprozesse zuordnen.

\subsubsection{Anpassbare Geschäftslogik}
Ein Unternehmen ist ein dynamisches Kontrukt, welches viele verschiedene
Prozesse beinhaltet. Einige dieser Prozesse können sich mit der Zeit wandeln.
Dementsprechend ist es wichtig, dass die Software dementsprechend anpassbar, \zB
über Parameter, ist ohne den Quellcode zu ändern.

\subsubsection{Große Anzahl von Nutzern}
Ein betrieblich genutzte Software wird, in den aller meisten Fällen, nicht nur
von einer einzigen Person \bzw Abteilung benutzt. Es ist deshalb umso wichtiger
, dass eine sollche Software sicherstellen kann, dass kein unbefugter Zugriff
auf Daten erfolgen kann.

Um das zu realisieren benutzen die meisten ERP-Systeme Gruppen- und
Einzelberechtigungen um einzelne Programmfunktionen aus und an zu schalten.

\subsubsection{Lokalisierung und Internationalisierung}
Um den länderübergreifenden Einsatz zu ermöglichen, muss die Software für die
verschiedenen Länder anpassbar sein. 

\begin{itemize}
  \item Sprache
  \item Währung
  \item Länderspezifische Steuersätze
  \item Einheiten 
\end{itemize}

\subsubsection{Behandlung monetärer Aspekte und Mehrwährungsaspekte}
Für verschiedene Länder müssen Geldbeträge in unterschiedlichen Währungen
verarbeitet werden. Dazu sind die verschiedenen Dinge aus Punkt fünf von
Relevanz um \zB zwischen verschiedenen Währungseinheiten umrechnen zu können
und somit den Geldumtausch zu gewährleisten.

\subsubsection{Transaktionsorientierung}
Einige Geschäftsprozesse müssen komplett durchlaufe. Sollte das aus
Irgendwelchen Gründen nicht funktionieren, so muss es Möglich sein den gesamten
Prozess zurück zu nehmen. 

Dies wird mittels Transaktionen realisiert. Ähnlich einer transaktion in eine
Datenbank kann man diese unabhängig von anderen Transaktionen ausführen und
zurück nehmen.

\subsubsection{Dauerhafte Speicherung von Daten}
Betriebsrelevante Daten müssen dauerhaft (d.h. persistent) und für andere
verfügbar gespeichert werden. 

Standardmäßig reicht dafür eine unternehmensinterne Datenbank, die von der
Software beschrieben und gelesen wird, völlih aus.

\subsubsection{Objektrelationale Abbildung}
Eine stärke von guter betrieblicher Software ist die Darstellung von Daten als
gekapseltes Objekt. Diese Objekt kann dann von der Software mit wenig Aufwand an
mehreren Stellen gleichzeitig verwendet werden.

typiuscherweiße wird dieses Verhalten mittels eines OR-Mappers (ORM)
implementiert und ist für den Benutzer der Software meistens irrelevant.


\subsection{Teilaufgabe B)}
\textbf{Welche drei Arten von Mustern gibt es? Beschreiben Sie jede Art kurz
(max. 2 Sätze).}

Muster werden in das Analysemuster, Unterstützungsmuster und Entwurfsmuster
unterteilt.

\begin{description}
  \item[Analysemuster:]
  Analysemuster werden in den Phasen vor dem Anwendungsentwurf im Rahmen der
  objektorientierten Analyse (ooA) genutzt.
  
  \item[Unterstützungsmuster:] 
  Unterstützungsmuster sind „an sich bereits nützlich“ und beschreiben, wie mit
  Analysemustern umgegangen werden soll (z.B. bei der Abbildung auf
  Entwurfsmuster) bzw. stellen wieder verwendbare Konzepte im späteren
  Designs dar.
  
  \item[Entwurfsmuster:] 
  Entwurfsmuster stellen bewährte Lösungsschablonen für wiederkehrende
  Entwurfsprobleme dar.
\end{description}

\subsection{Teilaufgabe C)}
\textbf{Welche Vorteile bzw. Nachteile/Einschränkungen hat das Analysemuster
Organisationsstruktur? Nennen Sie mindestens einen Vorteil und mindestens einen
Nachteil/eine Einschränkung.}

\begin{description}
  \item[Vorteile:] Verschiedene Hierarchie-Arten implementierbar (\zB
  Hierachrie eines Subunternehmen kann in den Entwurf mit rein genommen werden.
  
  \item[Nachteile:] Die aktuelle Tiefe der Hierarchie-Beziehung lässt sich nicht
  mehr ablesen.
\end{description}

\subsection{Teilaufgabe D)}
\textbf{Beschreiben Sie den Unterschied zwischen \gqq{Organisationsstruktur} und
\gqq{Verantwortlichkeit}.}
Während bei dem Analysemuster \gqq{Organisationsstruktur} nur Beziehungen
zwischen Organisationen über einen bestimmen Zeitraum nach modelliert werden
können, verallgemeinert das Analysemuster \gqq{Verantwortlichkeit} die Beziehung
zwischen \gqq{Objekten}.

Dadurch ist es möglich, komplexe Beziehungen zwischen Personen und bzw. oder
Organisationen abzubilden.

\subsection{Teilaufgabe E)}
\textbf{Erstellen Sie aus den beiden in der Vorlesung genannten Beispielen zum Analysemuster
„Organisationsstruktur“ (Folien 24 und 25) ein gemeinsames Klassendiagramm und
erweitern Sie dieses: Auch das Colaautomaten-Serviceteam 4711 ist dem Bostoner
Verkaufsbüro in Bezug auf das Bereichsmanagement unterstellt und berichtet an dieses.
Beim Produktsupport berichtet das Colaautomaten-Serviceteam 4711 an das
Colaautomaten-Servicecenter.}

Die nachfolgende Abbildung zeigt die beiden zusammengefügten Klassendiagramme
und die hinzugefügten Organisationen Colaautomaten-Serviceteam 4711 und
Colaautomaten-Servicecenter.

\includegraphicsKeepAspectRatio{Aufgabe1eueberarbeitet.png}{1}

\subsection{Teilaufgabe F)}
\textbf{Beschreiben Sie kurz die Grundidee beim Analysemuster \gqq{Wissensebene
der Verantwortlichkeit}.}

Die Grundidee bei der Wissensebene der Verantwortlichkeit ist die Zweiteilung
des Modells Organisationsstruktur in eine operationelle Ebene und eine
Wissensebene. Die operationelle Ebene enthält die Typen
\gqq{Verantwortlichkeit} und \gqq{Party} und deren Beziehungen. Die
Wissensebene hält die allgemeinen Regeln fest, die die zuständigen Strukturen bestimmen.

\clearpage
