\section{Aufgabe 1}

\subsection{Teilaufgabe A)}
\textbf{Nennen und erläutern Sie kurz die 10 Charakteristika betrieblicher
Anwendungen.}

\begin{description}
  \item[Strukturierte Daten:]
  Bei strukturierten Daten handelt es sich um eine Menge von Daten denen eine
  bestimmte, festgelegte Struktur zugrunde liegt.
  \item[Stamm- und Bewegungsdaten:]
  
  Stammdaten sind Informationen, die für die Dauer ihrer Existenz nicht \bzw nur selten geändert werden müssen. Sie haben
  in der Regel keinen Zeitbezug.
  Ein Beispiel für Stammdaten sind Kundendaten.
  
  Bewegungsdaten dagegen sind Informationen, die im Regelfall mit einer
  betriebswirtschaftlichen Transaktion im Zusammenhang stehen. Daher besitzen
  Bewegungsdaten auch einen Zeitbezug(\zB Erfassung eines Auftrags).
  \item[Anpassbare Geschäftslogik:]
  
  Da Unternehmen Ihre eigenen Abläufe abbilden wollen, muss eine Software anpassbar sein. Nur so kann diese Software von
  verschiedenen Benutzern eingesetzt werden.
  \item[Große Anzahl unterschiedlicher Benutzer:]
  
  Betriebswirtschaftliche Informationssysteme müssen meist eine große Anzahl an Nutzern verwalten. Damit
  keine ungewollten Änderungen an dem System durchgeführt werden, weist man
  einzelnen Gruppen oder Personen Rechte im Rahmen ihrer Aufgaben im Unternehmen
  zu.
  \item[Lokalisierung und Glokalisierung:]
  
  Um den länderübergreifenden Einsatz zu ermöglichen, muss die Software für die verschiendenen Länder anpassbar sein.
  Darunter fallen Dinge wie die Einstellung der Sprache, Währung und
  Einheitssysteme.
  \item[Behandlung monetärer Aspekte und Mehrwährungsaspekte:]
  
  Für verschiedene Länder müssen Geldbeträge in unterschiedlichen Währungen verarbeitet werden.
  Dazu sind die verschiedenen Dinge aus Punkt fünf von Relevanz um \zB zwischen
  verschiedenen Währungseinheiten umrechnen zu können und somit den Geldumtausch
  zu gewährleisten.
  \item[Transaktionsorientierung:]
  
  Einige betriebswirtschaftliche Vorgänge müssen als atomarer Vorgang
  durchgeführt werden. Diese Vorgänge nur komplett und als ganzes durchgeführt
  werden oder garnicht. Eine teilweise Ausführung ist nicht erlaubt und führt zur Inkonsistenz.
  \item[Dauerhafte Speicherung von Daten:]
  
  Der Anwendungszustand darf keinen Einfluss auf die dauerhafte Speicherung der Daten haben. Daraus
  resultiert, dass Daten auch dann zur Verfügung stehen, wenn die
  Betriebsbereitschaft der Anwendung nicht gegeben ist.
  \item[Objektrelationale Abbildung:]
  
  Wird ein Anwendungssystem für betriebswirtschaftliche Zwecke objektorientiert entwickelt, ihr aber eine
  relationale Datenbank zugrunde liegt, so muss eine Abbildung der beiden Welten
  stattfinden, dass sogenannte ORM(Objekt-Relationelle-Mapping).
\end{description}   


\subsection{Teilaufgabe B)}
\textbf{Welche drei Arten von Mustern gibt es? Beschreiben Sie jede Art kurz
(max. 2 Sätze).}


Muster werden in das Analysemuster, Unterstützungsmuster und Entwurfsmuster
unterteilt.
\begin{description}
  \item[Analysemuster:]
  
  Analysemuster werden in der Phase des Anwendungsentwurfs
  eingesetzt. Diese dienen der objektorientierten Analyse während der
  Projektplanung.
  \item[Unterstützungsmuster:] 
  
  Unterstützungsmuster beschreiben den Umgang mit
  Analysemustern. Ein Beispiel hierfür wäre die Abbildung eines Analysemusters
  auf einem Entwurfsmuster, das Ergebnis dieser Abbildung stellt also ein
  wiederverwendbares Konzept im späteren Design dar.
  \item[Entwurfsmuster:] 
  
  Entwurfsmuster stellen "`bewährte generische Lösung für
  häufig wiederkehrende Entwurfsprobleme"' dar. Dabei kann \ggfs zwischen
  allgemeinen und anwendungsspezifischen Analysemustern unterschieden werden.
\end{description}

\subsection{Teilaufgabe C)}
\textbf{Welche Vorteile bzw. Nachteile/Einschränkungen hat das Analysemuster
Organisationsstruktur? Nennen Sie mindestens einen Vorteil und mindestens einen
Nachteil/eine Einschränkung.}

\begin{description}
  \item[Vorteile:] Durch den Einsatz des Anaylemusters Organisationsstruktur
  kann ein Modell für einen Typ mehrere Hierarchien besitzen und lässt sich
  gleichzeitig über einen Zeitraum abbilden.
  \item[Nachteile:] Die aktuelle Art der Hierachiebeziehung lässt sich nicht
  mehr ablesen.
\end{description}


\subsection{Teilaufgabe D)}
\textbf{Beschreiben Sie den Unterschied zwischen „Organisationsstruktur“ und
„Verantwortlichkeit“.}
Während bei dem Analysemuster Organisationsstruktur nur Beziehungen zwischen
Organisationen über einen bestimmen Zeitraum nach definierten Regeln modelliert werden können,
verallgemeinert das Analysemuster die Verantwortlichkeit. Dadurch
ist es möglich, auch Beziehungen zwischen Personen und Organisationen zu
modellieren.


\subsection{Teilaufgabe E)}
\textbf{Erstellen Sie aus den beiden in der Vorlesung genannten Beispielen zum Analysemuster
„Organisationsstruktur“ (Folien 24 und 25) ein gemeinsames Klassendiagramm und
erweitern Sie dieses: Auch das Colaautomaten-Serviceteam 4711 ist dem Bostoner
Verkaufsbüro in Bezug auf das Bereichsmanagement unterstellt und berichtet an dieses.
Beim Produktsupport berichtet das Colaautomaten-Serviceteam 4711 an das
Colaautomaten-Servicecenter.}

Die nachfolgende Abbildung zeigt die beiden zusammengefügten Klassendiagramme
und die hinzugefügten Organisationen Colaautomaten-Serviceteam 4711 und
Colaautomaten-Servicecenter.

\includegraphicsKeepAspectRatio{Aufgabe1eueberarbeitet.png}{1}

\subsection{Teilaufgabe F)}
\textbf{Beschreiben Sie kurz die Grundidee beim Analysemuster „Wissensebene der
Verantwortlichkeit“.}

Die Grundidee bei der Wissensebene der Verantwortlichkeit ist die Zweiteilung
des Modells Organisationsstruktur in eine operationelle Ebene und eine
Wissensebene. Die operationelle Ebene enthält die Typen "`Verantwortlichkeit"'
und "`Party"' und deren Beziehungen. Die Wissensebene hält die allgemeinen
Regeln fest, die die zuständigen Strukturen bestimmen.

\clearpage 