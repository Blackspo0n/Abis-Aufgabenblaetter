\section{Aufgabe 1}

\subsection{Teilaufgabe A)}
\textbf{Nennen und erläutern Sie kurz die 10 Charakteristika betrieblicher
Anwendungen.}

1. Strukturierte Daten: Alle Daten haben eine festgelegte Struktur, z.B. enthält jeder
Auftragsdatensatz ein Feld KDNR (Kundennummer) an einer bestimmten Stelle.
2. Unterscheidung von Stamm- und Bewegungsdaten: Stammdaten sind weitgehend
zeitunabhängig (z.B. Kundendaten), während Bewegungsdaten zeitabhängig sind (z.B. ein
Auftrag enthält ein Auftragsdatum)
3. Anpassbare Geschäftslogik: In verschiedenen Unternehmen gibt es verschiedene Abläufe.
Soll dieselbe Software dort dennoch eingesetzt werden können, muss sie an die veränderten
Abläufe angepasst werden können.
4. Große Anzahl unterschiedlicher Benutzer: Mitarbeiter in verschiedenen Abteilungen (z.B.
Einkauf und Rechnungswesen) müssen mit demselben Softwarepaket arbeiten. Sie sollen
nur die Funktionen angeboten bekommen, die sie benötigen und auch keinen Zugriff auf
Daten außerhalb ihres Bereichs haben.
5. Lokalisierung und Internationalisierung: Die Software muss häufig in unterschiedlichen
Ländern mit unterschiedlichen Sprachen, Währungen, … eingesetzt werden können. Oft
müssen mehrere Sprachen oder Währungen gleichzeitig unterstützt werden.
6. Behandlung monetärer Aspekte und Mehrwährungsfähigkeit: Geldbeträge müssen in
unterschiedlichen Währungen verarbeitet werden können.
7. Transaktionsorientierung: Bestimmte betriebswirtschaftliche Vorgänge dürfen nicht
teilweise, sondern nur ganz oder gar nicht ausgeführt werden (z.B. Erfassung eines
Auftrags).
8. Dauerhafte Speicherung von Daten: Daten müssen unabhängig von der
Betriebsbereitschaft der Anwendung dauerhaft zur Verfügung stehen.
9. Objektrelationale Abbildung: Wenn man ein betriebswirtschaftliches Anwendungssystem
objektorientiert entwickelt, seine Daten aber in einer relationalen Datenbank speichern
möchte, muss eine Abbildung zwischen beiden Welten stattfinden.
10. Schichtenkonzept und Verteilung: Betriebswirtschaftliche Anwendungen laufen meist auf
mehreren Rechnern gleichzeitig (PCs der Benutzer und zentraler Server). Außerdem gibt es
mehrere Schichten bei der Softwareentwicklung (Benutzungsoberfläche, Anwendungslogik,
Datenhaltung).

\subsection{Teilaufgabe B)}
\textbf{Welche drei Arten von Mustern gibt es? Beschreiben Sie jede Art kurz
(max. 2 Sätze).}

Analysemuster werden in den Phasen vor dem eigentlichen Anwendungsentwurf im
Rahmen der objektorientierten Analyse (ooA) benutzt.
Entwurfsmuster stellen „bewährte generische Lösungen für häufig wiederkehrende
Entwurfsprobleme“ dar.
Unterstützungsmuster sind „an sich bereits nützlich“ und beschreiben, wie mit
Analysemustern umgegangen werden soll (z.B. bei der Abbildung auf Entwurfsmuster) bzw.
stellen wieder verwendbare Konzepte im späteren Designs dar.

\subsection{Teilaufgabe C)}
\textbf{Welche Vorteile bzw. Nachteile/Einschränkungen hat das Analysemuster
Organisationsstruktur? Nennen Sie mindestens einen Vorteil und mindestens einen
Nachteil/eine Einschränkung.}

Vorteil: Beliebige Hierarchien mit beliebig vielen Stufen lassen sich ohne Änderung abbilden.
Nachteil: Man kann im Modell nicht mehr erkennen, welche Hierarchiestufen es konkret gibt.


\subsection{Teilaufgabe D)}
\textbf{Beschreiben Sie den Unterschied zwischen „Organisationsstruktur“ und
„Verantwortlichkeit“.}


\subsection{Teilaufgabe E)}
\textbf{Erstellen Sie aus den beiden in der Vorlesung genannten Beispielen zum Analysemuster
„Organisationsstruktur“ (Folien 24 und 25) ein gemeinsames Klassendiagramm und
erweitern Sie dieses: Auch das Colaautomaten-Serviceteam 4711 ist dem Bostoner
Verkaufsbüro in Bezug auf das Bereichsmanagement unterstellt und berichtet an dieses.
Beim Produktsupport berichtet das Colaautomaten-Serviceteam 4711 an das
Colaautomaten-Servicecenter.}

\subsection{Teilaufgabe F)}
\textbf{Beschreiben Sie kurz die Grundidee beim Analysemuster „Wissensebene der
Verantwortlichkeit“.}


\clearpage 