\section{Aufgabe 2 (Implementierung eines Beispiels in Java)}
\textbf{Implementieren Sie eine sehr einfache Mini-Tabellenkalkulation:
Erstellen Sie eine Tabelle mit drei Spalten und fünf Zeilen. In der ersten Spalte sollen zu
Beginn die Zahlen 1 bis 5 ausgegeben werden. Der Benutzer kann hier aber auch eigene
Werte eingeben. In der zweiten Spalte sollen automatisch die Quadratwurzeln der Zahlen
aus der ersten Spalte stehen (am Anfang also 1, 1,414, 1,732, …). In der dritten Spalte
schließlich sollen die 3. Wurzeln (am Anfang also 1, 1,26…, 1,442…, … ) ausgegeben
werden.}

\textbf{Erläutern Sie im Quellcode kurz, welche Methoden Sie wie und warum
implementieren.}

Die Basis des TableModels bildet die Rumpfimplementierung, die uns in Moodle zur
Verfügung gestellt worden ist. Statt das Interface \gqq{TableModel} zu
implementieren, besteht in Swing die Möglichkeit von der abstrakten Klasse 
\gqq{AbstractTableModel} zu erben. Der Vorteil besteht darin, dass weniger
Methoden implementieren werden müssen, als wenn wir direkt das Interface
implementieren.

Folgende Methoden müssen implementiert werden, wenn von
\gqq{AbstractTableModel} abgeleitet worden ist:

\begin{description}
\item[getRowCount] Gibt die Anzahl der Zeilen us
\item[getColumnCount] Gibt die Anzahl der Spalten aus
\item[getColumnName] Gibt den Namen der Spalte mittels eines Index aus
\item[getValueAt] Gibt an, welchen Wert eine Zelle besitzt.
\end{description}

Unser implementiertes Model sieht folgendermaßen aus:
\begin{lstlisting}[language=java, style=java, caption={WurzelTableModel.java},
label={lst:lst1}]
package tableModel;

import javax.swing.table.*;

class WurzelTableModel extends AbstractTableModel {
	final String[ ] columnNames = {"X", "Wurzel", "3. Wurzel"};
	int[] ersteSpalte = {1,2,3,4,5};
	private static final long serialVersionUID = 930096348695238927L;
	
	/**
	 * Da wir nur 3 Saplten haben, ist hier hardcoded eine 3
	 */
	public int getColumnCount() {
		return 3;
	}
    /**
	 * Zurückgegeben wird hier die Länge des Arrays ersteSpalte.
	 */
	public int getRowCount() {
		return ersteSpalte.length;
	}


	/**
	 * Mit dieser Funktion wird der Name der Spalte zurückgegeben.
	 * Zugegriffen wird auf das Array Columnname
	 */
	public String getColumnName(int columnIndex) {
		return columnNames[columnIndex];
	}
	/**
	 * Diese Funktion errechnet unsere Werte und gibt die für jede einzelne
	 * Position entsprechend aus
	 */
	public Object getValueAt(int rowIndex, int columnIndex) {
		switch (columnIndex) {
		case 2:
			return Math.pow(ersteSpalte[rowIndex], (1.0/3));
		case 1:
			return Math.sqrt(ersteSpalte[rowIndex]);
		case 0:
		default:
			return ersteSpalte[rowIndex];
		}
	}
}
\end{lstlisting}

Unser Ausgabe sieht dann folgendermaßen aus:

\includegraphicsKeepAspectRatio{img1.png}{1}

\clearpage
