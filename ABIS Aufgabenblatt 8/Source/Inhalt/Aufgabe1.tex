\section{Aufgabe 1 (Konzept und Aufbau des MVC-Musters)}

\subsection{Teilaufgabe A)}
\textbf{Aus welchen Komponenten besteht das MVC-Muster und welche Aufgabe hat die
jeweilige Komponente?}

\subsubsection{Model}
Die erste Ebene dieses Musters ist die Modelkomponente. Sie representiert
Daten in der Anwendung und implementiert das Muster "`Observer"' um Änderungen
der View mit zu teilen.

Die Aufgabe des Models ist es Daten unabhängig von dem Controller und der View
zu halten. Durch diese Eigenschaft kann es zu einem Modell mehrere Controller
und mehrere Views geben.

\subsubsection{View}
Die nächste Komponente ist die View. Diese representiert die
Interaktionsschicht, in den meisten Fällen eine GUI, für den Benutzer.

Sie hält jeweils Referenzen zu den Models und dem Controller.

Somit hat die View die Aufgabe Daten aus dem Model anzuzeigen. Dies kann auf
unterschiedliche Art, \zB als Tabelle, geschehen.

\subsubsection{Controller}

Die Logikschicht bildet der Controller. In dieser Komponente wird auf
Benutzereingaben reagiert, wertet diese aus und agiert entsprechend.

\clearpage
\subsection{Teilaufgabe B)}
\textbf{Welche Muster finden sich im zusammengesetzen MVC-Muster wieder? Nennen und
beschreiben Sie (kurz!) drei enthaltene Muster.}

\subsubsection{Strategie}
Wird verwendet in: \textbf{View} \& \textbf{Controller}

Eine Strategie beschreibt, was ein Objekt zutun hat ohne es aber direkt zu
implementieren. 

In unserem Konzept bedeutet das, dass die View weiß, dass es ein Controller gibt
und was dieser kann, aber nicht weiß, was dieser genau macht.

\subsubsection{Composite}
Wird verwendet in: \textbf{View}

Mittels dem Composite ist es Ganz-teil Herachrichen zu implementieren.

Im Kontext des Musters kann eine View aus mehreren Komponenten bestehen.

\subsubsection{Observer}
Wird verwendet in: \textbf{Modell}

Das Muster Observer informiert registrierte Klassen über veränderungen an den
eigenen Eigenschaften.

Im Kontext des MVC Konzept informiert das Modell so die View, dass sich daten
geändert haben.

\subsection{Teilaufgabe C)}
\textbf{In den Java Swing-Klassen wird das MVC-Muster nicht 1:1 implementiert. Welche
Veränderung wird dort vorgenommen und warum geschieht dies?}

Die Komponenten in Swing werden in einer Delegaten zusammengefasst.
Dieses Design wurde aus Performance-Gründen so gewählt, da beim MVC-Muster
erhöhte Aufrufbeziehungen zwischen View und Controller bestehen und so versucht
wird verschatelte Aufrufe zu minimieren.

\clearpage