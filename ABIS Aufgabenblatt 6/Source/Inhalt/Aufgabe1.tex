\section{Aufgabe 1 (Analysemuster)}

\subsection{Teilaufgabe A)}
\textbf{Die Prinzipien der Muster Quantität und Umwandlungsverhältnis kann auch auf
Währungen angewendet werden. Beschreiben Sie wie dies erfolgen kann? Welche
Erweiterungen sind eventuell erforderlich?}

Die Geldangabe in einer Währung ist eine Quantität, denn der Betrag in \zB Euro
hat einen Wert in Euro-Einheiten. 

Zum Beispiel \eur{13,00} haben den Wert von 13, der Einheit Euro.

Umwandlungsverhältnisse existieren innerhalb einer Währung, \zB zwischen Euro
und Cent, dabei entsprichen \eur{13,00} genau 1300 Cent. 

Umwandlungsverhältnisse existieren aber auch zwischen Währungen, z.B.
entspricht \eur{1,00} in etwa 1,12 US-Dollar. 

Hierbei muss jedoch eine Zeitangabe hinzugefügt werden, da sich der
Umrechnungskurs zwischen Währungen ständig ändert.

\subsection{Teilaufgabe B)}
\textbf{Was bedeutet die Unterscheidung zwischen Wissensebene und operationeller Ebene
beim Muster \gqq{Messung}?}

Die beiden Ebenen unterscheiden sich dahingehen, dass die Wissensebene
festhält, was gemessen werden kann, \zB Größe, Gewicht. Die operationale Ebene
hält konkrete Messungen, \zB einer Größe, fest, die gemessen werden kann.
\clearpage
