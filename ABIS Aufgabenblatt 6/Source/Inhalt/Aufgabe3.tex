\section{Aufgabe 3 (Erzeugungsmuster)}

\textbf{Nehmen Sie an, dass Sie ein Java-Programm erstellen müssen, in dem Sie eine Klasse
„Fahrzeuge“ benutzen wollen. Allerdings werden später als Objekte unterschiedliche Arten
von Fahrzeugen, nämlich \gqq{Fahrräder} und \gqq{Autos} vorkommen, die auch zum
Teil voneinander abweichende Attribute und unterschiedliche Realisierungen derselben
Methoden haben werden. Es ist auch damit zu rechnen, dass später weitere
Fahrzeugklassen (\zB Tretroller) hinzukommen. Erstellen Sie eine einfache
Implementierung, in der die Muster \gqq{Abstrakte Fabrik} und \gqq{Singleton}
verwendet werden (dabei soll \gqq{Singleton} nicht im Zusammenhang mit der
Klasse \gqq{Fahrzeuge} verwendet werden).}

\textbf{Abzugeben sind der Quelltext und ein Beispielablauf, der ein Fahrrad
und ein Auto zeigt.}

\begin{lstlisting}[language=java, style=java, caption={Main.java},
label={lst:lst1}]
package Main;
import Entities.Fahrzeug;
import Factory.AbstractFactory;
import Factory.FactoryCreator;

public class Main {
	public static void main(String[] args) {
				// Zum verdeutlichen Variablen explizit deklariert
		AbstractFactory factory;
		Fahrzeug fahrzeug;
		
		factory = FactoryCreator.getInstance().createfactory("Auto");
		fahrzeug = factory.create(new String[]{"Silber", "VW Passat", "106"});
		
		fahrzeug.beschreibung();
		fahrzeug.beschleunigen();
		fahrzeug.hupe();
		fahrzeug.bremsen();
		
		factory = FactoryCreator.getInstance().createfactory("Fahrrad");
		fahrzeug = factory.create(new String[]{"Grün","27", "Scheibenbremsen"});
		
		fahrzeug.beschreibung();
		fahrzeug.beschleunigen();
		fahrzeug.hupe();
		fahrzeug.bremsen();
		
	}
}
\end{lstlisting}


\begin{lstlisting}[language=java, style=java, caption={Fahrzeug.java},
label={lst:lst2}]
package Entities;

public interface Fahrzeug {
	public void beschreibung();
	public void bremsen();
	public void beschleunigen();
	public void hupe();
}
\end{lstlisting}


\begin{lstlisting}[language=java, style=java, caption={Fahrrad.java},
label={lst:lst3}]
package Entities;

public class Fahrrad implements Fahrzeug {
	private String Farbe;
	private String anzahlDerGaenge;
	private String BremsenArt;
	
	public Fahrrad(String farbe, String anzahlDerGaenge, String bremsenArt) {
		setFarbe(farbe);
		setAnzahlDerGaenge(anzahlDerGaenge);
		setBremsenArt(bremsenArt);
	}
	
	public String getFarbe() {
		return Farbe;
	}

	public void setFarbe(String farbe) {
		Farbe = farbe;
	}

	public String getAnzahlDerGaenge() {
		return anzahlDerGaenge;
	}

	public void setAnzahlDerGaenge(String anzahlDerGaenge) {
		this.anzahlDerGaenge = anzahlDerGaenge;
	}

	public String getBremsenArt() {
		return BremsenArt;
	}

	public void setBremsenArt(String bremsenArt) {
		BremsenArt = bremsenArt;
	}

	@Override
	public void bremsen() {
		System.out.println("[Fahrrad] Der Radfahrer bremst!");
	}

	@Override
	public void beschleunigen() {
		System.out.println("[Fahrrad] Beschleunigen. Immer Flotter und immer weiter, so der Radfahrer heiter!");
	}

	@Override
	public void beschreibung() {
		System.out.println("[Fahrrad] Farbe = " + getFarbe() + ", BremsenArt = " + getBremsenArt() + ", Anzahl der Gänge = " + getAnzahlDerGaenge());
	}

	@Override
	public void hupe() {
		System.out.println("[Fahrrad] Ringring");	
	}
}
\end{lstlisting}


\begin{lstlisting}[language=java, style=java, caption={Auto.java},
label={lst:lst4}]
package Entities;

public class Auto implements Fahrzeug {
	private String Farbe;
	private String Marke;
	private int Leistung;
	
	public Auto(String Farbe, String Marke, int Leistung) {
		setFarbe(Farbe);
		setMarke(Marke);
		setLeistung(Leistung);
	}
	
	public String getFarbe() {
		return Farbe;
	}

	public void setFarbe(String farbe) {
		Farbe = farbe;
	}

	public String getMarke() {
		return Marke;
	}

	public void setMarke(String marke) {
		Marke = marke;
	}

	public int getLeistung() {
		return Leistung;
	}

	public void setLeistung(int leistung) {
		Leistung = leistung;
	}

	@Override
	public void bremsen() {
		System.out.println("[Auto] Die Reifen quitschen laut während des Bremsens.");
	}

	@Override
	public void beschleunigen() {
		System.out.println("[Auto] Der Fahrer gibt Vollgas");
	}

	@Override
	public void beschreibung() {
		System.out.println("[Auto] Farbe = " + getFarbe() + ", Marke = " + getMarke() + ", Leistung = " + getLeistung());
		
	}

	@Override
	public void hupe() {
		System.out.println("[Auto] Moooeep");
		
	}
}
\end{lstlisting}


\begin{lstlisting}[language=java, style=java, caption={AbstractFactory.java},
label={lst:lst5}]
package Factory;

import Entities.Fahrzeug;

public abstract class AbstractFactory {
	public abstract Fahrzeug create(String[] FactoryParameters);
}
\end{lstlisting}

\begin{lstlisting}[language=java, style=java, caption={FahrradFactory.java},
label={lst:lst6}]
package Factory;

import Entities.Fahrrad;
import Entities.Fahrzeug;

public class FahrradFactory extends AbstractFactory {
	@Override
	public Fahrzeug create(String[] FactoryParameters) {
		if(FactoryParameters.length != 3) return null;
		
		return new Fahrrad(FactoryParameters[0], FactoryParameters[1], FactoryParameters[2]);
	}
}
\end{lstlisting}

\begin{lstlisting}[language=java, style=java, caption={AutoFactory.java},
label={lst:lst7}]
package Factory;

import Entities.Auto;
import Entities.Fahrzeug;

public class AutoFactory extends AbstractFactory {
	@Override
	public Fahrzeug create(String[] FactoryParameters) {
		if(FactoryParameters.length != 3) return null;
			
		return new Auto(FactoryParameters[0], FactoryParameters[1], Integer.parseInt(FactoryParameters[2]));		
	}
}
\end{lstlisting}

\begin{lstlisting}[language=java, style=java,
caption={FactoryCreator.java}, label={lst:lst8}]
package Factory;

public class FactoryCreator {
	// Singleton Pattern
	public static FactoryCreator instance;
	
	private FactoryCreator() {
		// Leerer privater Konstruktor 
	};
	
	public static FactoryCreator getInstance() {
		if(instance == null) {
			instance = new FactoryCreator();
		}
		
		return instance;
	}
	
	public AbstractFactory createfactory(String FactoryType) {
		if(FactoryType.equals("Fahrrad")) {
			return new FahrradFactory();
		}
		
		if(FactoryType.equals("Auto")) {
			return new AutoFactory();
		}
		
		return null;
	}
}
\end{lstlisting}

Ausgabe der Konsole:

\includegraphicsKeepAspectRatio{img2.png}{1}

\clearpage
