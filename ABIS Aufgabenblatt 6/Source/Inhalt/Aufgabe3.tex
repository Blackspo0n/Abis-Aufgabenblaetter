\section{Aufgabe 3 (Erzeugungsmuster)}

\textbf{Nehmen Sie an, dass Sie ein Java-Programm erstellen müssen, in dem Sie eine Klasse
„Fahrzeuge“ benutzen wollen. Allerdings werden später als Objekte unterschiedliche Arten
von Fahrzeugen, nämlich „Fahrräder“, „Motorräder“ und „Autos“ vorkommen, die auch zum
Teil voneinander abweichende Attribute und unterschiedliche Realisierungen derselben
Methoden haben werden. Es ist auch damit zu rechnen, dass später weitere
Fahrzeugklassen (z.B. Tretroller) hinzukommen. Erstellen Sie eine einfache
Implementierung, in der die Muster „Abstrakte Fabrik“ und „Singleton“ verwendet werden
(dabei soll „Singleton“ nicht im Zusammenhang mit der Klasse „Fahrzeuge“ verwendet
werden).
Abzugeben sind der Quelltext und ein Beispielablauf!}

\clearpage
