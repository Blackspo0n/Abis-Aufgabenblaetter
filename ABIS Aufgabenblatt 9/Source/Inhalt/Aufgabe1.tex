\section{Aufgabe 1 (Internationalisierung und Lokalisierung)}

\subsection{Teilaufgabe A)}
\textbf{Erklären Sie die Begriffe „Internationalisierung“ und „Lokalisierung“. Wo gibt es
Gemeinsamkeiten? Was sind Unterschiede?}

\subsubsection{L10N}
Unter Lokalisierung (Localisation L10N) versteht man die Anpassung einer
Anwendung an die Gegebenheiten einer Region, die sich hinsichtlich
verschiedener Kriterien von anderen Regionen unterscheiden kann.

\subsubsection{I18N}
Internationalisierung (Internationalisation I18N) bedeutet primär die Herstellung
von Lokalisierbarkeit. Es geht darum, Anwendungen so zu erstellen, dass sie
effizient und ohne nachträgliche Anpassungen im Quellcode für unterschiedliche
Länder und Regionen einsetzbar gemacht werden können.
Oft wird Internationalisierung weitergehend definiert; meist wird die gleichzeitige
Nutzbarkeit unterschiedlicher nationaler Aspekte angestrebt. Für Währungen
bedeutet dies z.B., dass das System mit mehreren Währungen parallel arbeiten
und die Währung ineinander umwandeln kann. 

\subsubsection{Unterschiede}
I18N beschreibt das Vorgehen eine Software so zu erstellen, dass L10N auf dieser
Basis angewendet werden kann.

\subsection{Teilaufgabe B)}
\textbf{Was ist ein Ressourcenbündel? Warum ist es zweckmäßig, Zeichenketten (Strings) aus
dem Quelltext in ein Ressourcenbündel auszulagern?}

Die \gqq{ResourceBundle}-Objekte sind spezielle Assoziativspeicher, die alle
programmrelevanten Texte und Informationen für ein spezielles Land
repräsentieren. 

Die Zweckmäßigkeit Strings auszulagern liegt darin, dass die Software zur
Laufzeit, durch entsprechende Logik, entscheiden kann, welche Zeichenkette
geladen wird.

Weiter können die Ressourcen unabhängig von dem eigentlichen Quellcode verändert
werden. Somit kann ein Übersetzungsteam, welches nicht entwickeln kann, ein
Software-Projekt in eine beliebige Sprache übersetzen, ohne je den eigentlichen
Quellcode gesehen zu haben.


\subsection{Teilaufgabe C)}
\textbf{Erweitern Sie das Beispiel aus der Vorlesung um eine gemäß der Lokalisierung korrekt
formatierte Zahl mit Nachkommastellen (z.B. 1234,56 für DE 1.234,56 und für EN
1,234.56). Geben Sie den Quelltext an (Ausschnitt reicht) und erstellen Sie ein
Ablaufprotokoll.}

\begin{lstlisting}[language=java, style=java, caption={NumberFormat},
label={lst:lst1}]
	[...]
        ResourceBundle rb = ResourceBundle.getBundle("myBundle",myLocale);
        System.out.print(rb.getString("format_number:_"));
        
        NumberFormat nf = NumberFormat.getInstance(myLocale);
        System.out.println(nf.format(1234.56));       
    [...]
\end{lstlisting}
\begin{figure}[htb]
\begin{center}
\includegraphicsKeepAspectRatio{Auswahl_004.png}{0.8}
\caption{Ausgabe Deutsch}
\end{center}
\end{figure}

\begin{figure}[htb]
\begin{center}
\includegraphicsKeepAspectRatio{Auswahl_005.png}{0.8}
\caption{Ausgabe Englisch}
\end{center}
\end{figure}

\clearpage
\subsection{Teilaufgabe D)}
\textbf{Erweitern Sie das Lokalisierungsbeispiel aus der Vorlesung (siehe Moodle) um ein
Ressourcenbündel Ihrer Wahl (z.B. Französisch). Testen Sie das Programm mit Ihrem
neuen Ressourcenbündel und erstellen Sie ein Ablaufprotokoll.}

Es wurde ein türkisches Sprachparket erstellt.

\begin{figure}[htb]
\begin{center}
\includegraphicsKeepAspectRatio{Auswahl_006.png}{0.8}
\caption{Ausgabe Türkisch}
\end{center}
\end{figure}

\subsection{Teilaufgabe E)}
\textbf{Geben Sie ein Lokalisierungsproblem an, das sich wahrscheinlich nicht oder nur schwer
über Ressourcenbündel und Locales lösen lässt.}

Ein Beispiel, was sich nicht ohne weiteres über Ressourcenbündel realisieren
lässt, sind Daten, die in einer Datenbank abgelegt werden. Da diese Daten
oftmals verschiedene Texte beinhalten können Ressourcenbündel diese schlecht bis
überhaupt nicht \gqq{matchen}. 

Um diese Daten trotzdem in verschiedener Sprache verfügbar zu machen muss über
eine Lösung in der Datenbank nachgedacht werden. So könnte man Beispielsweiße
für die Artikelstammdaten einfach eine Übersetzungstabelle anlegen, die die
Übersetzungen pro Artikel beinhalten.

\clearpage