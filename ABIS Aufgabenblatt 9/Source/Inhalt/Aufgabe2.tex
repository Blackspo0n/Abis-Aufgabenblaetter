\section{Aufgabe 2 (Framework)}
\subsection{Aufgabenstellung}
Entwickeln Sie ein einfaches Framework zur Implementierung des „DAO“-Entwurfsmusters.
Welche Art von Framework (White Box, Black Box oder Mischform) entwickeln Sie?

Nutzen Sie das selbst entwickelte Framework für die Speicherung von Objekten der Klasse
Artikel. Ein Artikel hat als Schüssel eine Artikelnummer vom Typ Long, eine Bezeichnung
und einen Preis.

Besonders gut ist Ihr Framework, wenn Sie dort auf die Klasse „Object“ verzichten können,
d.h. wenn Sie Ihr Framework korrekt mit einem Typparameter versehen und diesen auch
nutzen. (+5\%).

Abzugeben ist:
\begin{itemize}
  \item Ein kurzer Text, der den Aufbau Ihres Frameworks beschreibt (10\%)
  \item Der komplette Quelltext Ihres Frameworks (25\%)
  \item Ein Klassendiagramm, dass die Nutzung des Frameworks beim Speichern von
Objekten Ihrer Klasse „Artikel“ zeigt. (10\%)
  \item Der Quellcode Ihres Anwendungsbeispiels und ein Beispielablauf.(25 \%)
\end{itemize}

\subsection{Beschreibung}
\subsection{Quelltext des Frameworks}
\begin{lstlisting}[language=java, style=java, caption={ArticleDAO.java},
label={lst:lst1}]
package dao;

import java.sql.*;

import entity.Article;
import framework.AbstractDAO;

public class ArticleDAO extends AbstractDAO<Article> {

	static ArticleDAO instance;
	
	private ArticleDAO () {}
	
	public static ArticleDAO getInstance() {
		if(instance == null) {
			instance = new ArticleDAO();
		}
		
		return instance;
	}
	
	public String getInsertStatement() {
		return "INSERT INTO Article VALUES(?, ?, ?)";
	}

	public String getUpdateStatement() {
		return "UPDATE Article SET description = ?, price = ? WHERE articleNumber = ?";
	}

	public String getDeleteStatement() {
		return "DELETE FROM Article WHERE articleNumber = ?";
	}

	public String getFindStatement() {
		return "SELECT articleNumber, description, price FROM Article WHERE articleNumber=?";
	}

	public Article doLoad(ResultSet rs) throws SQLException {
		Long articleNumber = new Long(rs.getLong(1));
		String description = rs.getString(2);
		double price = rs.getDouble(3);
		
		Article result = new Article(articleNumber, description, price);
		return result;
	}

	public PreparedStatement doUpdate(Article entity, PreparedStatement stmnt) throws SQLException {
        stmnt.setLong(3, entity.getArticleNumber());
        stmnt.setString(1, entity.getDescription());
        stmnt.setDouble(2, entity.getPrice());
        
		return stmnt;
	}

	public PreparedStatement doInsert(Article entity, PreparedStatement stmnt) throws SQLException {
        stmnt.setLong(1, entity.getArticleNumber());
        stmnt.setString(2, entity.getDescription());
        stmnt.setDouble(3, entity.getPrice());
        
		return stmnt;
	}
}
\end{lstlisting}



\begin{lstlisting}[language=java, style=java, caption={Article.java},
label={lst:lst6}]
package entity;

import java.sql.SQLException;
import dao.ArticleDAO;
import framework.Entity;

public class Article implements Entity {
	private Long articleNumber;
	private String description;
	private double price;
	
	private ArticleDAO articleDAO = ArticleDAO.getInstance();

	public Article(Long articleNumber, String description, double price) throws SQLException {
		setArticleNumber(articleNumber);
		setDescription(description);
		setPrice(price);
		
		articleDAO.insert(this);
	}

	public Long getArticleNumber() {
		return articleNumber;
	}

	public void setArticleNumber(Long articleNumber) throws SQLException {
		this.articleNumber = articleNumber;
		articleDAO.update(this);
	}

	public String getDescription() {
		return description;
	}

	public void setDescription(String description) throws SQLException {
		this.description = description;
		articleDAO.update(this);
	}

	public double getPrice() {
		return price;
	}

	public void setPrice(double price) throws SQLException {
		this.price = price;
		articleDAO.update(this);
	}

	public ArticleDAO getArticleDAO() {
		return articleDAO;
	}

	public void setArticleDAO(ArticleDAO articleDAO) throws SQLException {
		this.articleDAO = articleDAO;
		articleDAO.update(this);
	}

	public static Article read(Long articlenumber) throws SQLException {
		return ArticleDAO.getInstance().read(articlenumber);
	}

	public static void delete(Article article) throws SQLException {
		ArticleDAO.getInstance().delete(article);
	}

	@Override
	public Long getPrimaryKey() {
		return getArticleNumber();
	}
}
\end{lstlisting}

\begin{lstlisting}[language=java, style=java, caption={AbstractDAO.java},
label={lst:lst3}]
package framework;

import java.sql.Connection;
import java.sql.PreparedStatement;
import java.sql.ResultSet;
import java.sql.SQLException;
import java.util.HashMap;

public abstract class AbstractDAO<T extends Entity> {

    private HashMap<Long, T> cache = new HashMap<Long, T>();
    public Connection connection = null;
	
    public AbstractDAO() {
		connection = Database.getConnection();
	}
	
	public abstract String getInsertStatement();
	public abstract String getUpdateStatement();
	public abstract String getDeleteStatement();
	public abstract String getFindStatement();
	public abstract T doLoad(ResultSet rs) throws SQLException;
	public abstract PreparedStatement doUpdate(T entity, PreparedStatement stmnt) throws SQLException;
	public abstract PreparedStatement doInsert(T entity, PreparedStatement stmnt) throws SQLException;
	
	public T read(Long primaryKey) throws SQLException {

        if(cache.containsKey(primaryKey)) return cache.get(primaryKey);
		        
		PreparedStatement findStatement = null;
        findStatement = connection.prepareStatement(getFindStatement());
        findStatement.setLong(1, primaryKey);
        ResultSet rs = findStatement.executeQuery();
        rs.next();
        
        T result = load(rs);
    
		cleanUp();
		
	    return result;
	}
	
	public T load(ResultSet rs) throws SQLException {
        Long primaryKey = new Long(rs.getLong(1));
        
        if(cache.containsKey(primaryKey)) return cache.get(primaryKey);
        
        T loadedobj = doLoad(rs);
        cache.put(primaryKey, loadedobj);
        
        return loadedobj;
	}
	
	public void update(T Entity) throws SQLException {
		PreparedStatement updateStatement = null;
		
        updateStatement = connection.prepareStatement(getUpdateStatement());
        updateStatement = doUpdate(Entity, updateStatement);
        updateStatement.execute();
        
        cache.put(Entity.getPrimaryKey(), Entity);
	}

	public Long insert(T Entity) throws SQLException {
		PreparedStatement insertStatement = null;
		
        insertStatement = connection.prepareStatement(getInsertStatement());
        insertStatement = doInsert(Entity, insertStatement);
        insertStatement.execute();
        
        cache.put(Entity.getPrimaryKey(), Entity);
        
        return Entity.getPrimaryKey();
	}

	public void delete(T Entity) throws SQLException {
        PreparedStatement deleteStatement = null;
        
        deleteStatement = connection.prepareStatement(getDeleteStatement());
        deleteStatement.setLong(1, Entity.getPrimaryKey());
        deleteStatement.execute();
        
        cache.remove(Entity.getPrimaryKey());
        
	}
	
	public void cleanUp() {
		cache.clear();
	}
}
\end{lstlisting}

\begin{lstlisting}[language=java, style=java, caption={Database.java},
label={lst:lst4}]
package framework;

import java.sql.*;

public class Database {
	static Connection getConnection() {
        try {
            Class.forName("com.mysql.jdbc.Driver");
            return DriverManager.getConnection("jdbc:mysql://localhost/xdb", "root", "");
        } catch (Exception ex) {
            ex.printStackTrace();
        }
        return null;
	}
}
\end{lstlisting}

\begin{lstlisting}[language=java, style=java, caption={Entity.java},
label={lst:lst5}]
package framework;

public interface Entity {
	public abstract Long getPrimaryKey();
}
\end{lstlisting}
\clearpage

\subsection{Klassendiagram}

\begin{figure}[htb]
	\centering
	\includegraphicsKeepAspectRatio{uml.png}{1}
	\caption{UML-Digramm}
\end{figure}

\subsection{Quelltext des Anwendungsbeispiel}
\begin{lstlisting}[language=java, style=java, caption={ArticleDAOTest.java},
label={lst:lst2}]
package dao;

import java.sql.SQLException;

import entity.Article;

public class ArticleDAOTest {
	public static void main(String[] args) {
		System.out.println("############ Framework Test ###########");
		
		System.out.println("Creating des Article nimst fahrt auf:");
        Article article;
		try {
			article = new Article(1331l, "Mega geiler Laptop mit Pentium 4 Prozessor", 2999.99);
		} catch (SQLException e1) {
			e1.printStackTrace();
		}
        
        System.out.println("Putte lokale Variable at NULL und gette Article back:");
        article = null;
        try {
			article = Article.read(1331l);
		} catch (SQLException e1) {
			e1.printStackTrace();
		}
        
        System.out.println("Der Artikel bimst vong der Presistensigkeit her follewed " + article.getArticleNumber() + " "
                + article.getDescription()+", "+article.getPrice());
                
        System.out.println("Refreshe Article. Price auf 3099,99 am been");
        try {
			article.setPrice(3099.99);
		} catch (SQLException e1) {
			e1.printStackTrace();
		}
        
        article = null;
        try {
			article = Article.read(1331l);
		} catch (SQLException e1) {
			e1.printStackTrace();
		}
        System.out.println("Article bimst jetzt mit Price " + article.getPrice());

 
        try {
			Article.delete(article);
		} catch (SQLException e1) {
			e1.printStackTrace();
		}
        article = null;
        
        System.out.println("Trye den Article nach destruction erneud zu readen: ");
        try {
            article = Article.read(1331l);
            System.out.println(article);
        }
        catch (Exception e) {
        	System.err.println("readen des Articles bimst fehlgeschlagen");
		}
	}
}
\end{lstlisting}
\clearpage
