\section{Aufgabe 2 (Erstellung eines DTD-Dokuments)}


\subsection{Teilaufgabe a)}
\textbf{Erstellen Sie eine DTD für die XML-Datei aus Aufgabe 1.}

\begin{lstlisting}[language=xml, caption={XML-Datei},
label={lst:xml}]
<?xml version="1.0" encoding="UTF-8"?>
<!ELEMENT KFZ (Auto*,Motorrad*)>
<!ELEMENT Auto (Kennzeichen,Halter,Typ)>
<!ELEMENT Motorrad (Kennzeichen,Halter)>
<!ELEMENT Kennzeichen (#PCDATA)>
<!ELEMENT Halter (Vorname+,Nachname)>
<!ELEMENT Vorname (#PCDATA)>
<!ELEMENT Nachname (#PCDATA)>
<!ELEMENT Typ (#PCDATA)>
\end{lstlisting}

\subsection{Teilaufgabeb b)}
\textbf{Bauen Sie Ihre XML-Datei so um, dass die DTD aus Teilaufgabe a benutzt
wird.}
\begin{lstlisting}[language=xml, caption={XML-Datei},
label={lst:xml}]
<?xml version="1.0"?>
<!DOCTYPE KFZ SYSTEM "KFZLIST.dtd">
<KFZ>
	[...] 	
</KFZ>
\end{lstlisting}

\subsection{Teilaufgabe c)}
\textbf{Was müssen Sie tun, damit Ihre DTD auch von anderen benutzt werden
kann?}

Das Schema muss auf einen Webserver verfügbargemacht werden. Weiter muss man
anschließend eine leicht abgewandelte Schreibweiße für das Einbetten des Schemas
verwenden:
\begin{lstlisting}[language=xml, caption={XML-Datei},
label={lst:xml}]
<?xml version="1.0"?>
<!DOCTYPE KFZ PUBLIC 'KFZ' 'https://example.com/xmlDTD/kfzlist.dtd'>
<KFZ>
	[...]
</KFZ>
\end{lstlisting}

\subsection{Teilaufgabe d)}
\textbf{Wandeln Sie Ihre Lösung aus a und b so um, dass Vorname und Nachname als Attribut
eines anderen Elements auftreten.}

\textbf{Schema:}
\begin{lstlisting}[language=xml, caption={XML-Datei},
label={lst:xml}]
<?xml version="1.0" encoding="UTF-8"?>
<!ELEMENT KFZ (Auto*,Motorrad*)>
<!ELEMENT Auto (Kennzeichen,Halter,Typ)>
<!ELEMENT Motorrad (Kennzeichen,Halter)>
<!ELEMENT Kennzeichen (#PCDATA)>
<!ELEMENT Halter EMPTY>
<!ATTLIST Halter Vorname CDATA #REQUIRED Nachname CDATA #REQUIRED>
<!ELEMENT Typ (#PCDATA)>
\end{lstlisting}

\textbf{XML-Struktur:}
\begin{lstlisting}[language=xml, caption={XML-Datei},
label={lst:xml}]
<?xml version="1.0"?>
<!DOCTYPE KFZ PUBLIC 'KFZ' 'https://example.com/xmlDTD/kfzlist.dtd' >
<KFZ>
	<Auto>
		<Kennzeichen>DN-WM10</Kennzeichen>
		<Halter Vorname="Mermett" Nachname="Mueller"/>
		<Typ>Limousine</Typ>
	</Auto>
</KFZ>
\end{lstlisting}

\clearpage 